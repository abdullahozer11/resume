%-------------------------
% CV in Latex
% Auteur : Abdullah Ozer
% License : MIT
%------------------------

\documentclass[letterpaper,10pt]{article}

\usepackage{bold-extra}
\usepackage{latexsym}
\usepackage[empty]{fullpage}
\usepackage{titlesec}
\usepackage{marvosym}
\usepackage[usenames,dvipsnames]{color}
\usepackage{verbatim}
\usepackage{enumitem}
\usepackage[hidelinks]{hyperref}
\usepackage{fancyhdr}
\usepackage{graphicx}
\usepackage[english]{babel}
\usepackage{tabularx}
\usepackage{hyphenat}
\usepackage{fontawesome}
\usepackage{tikz}
\usepackage{graphicx}
\usepackage{xcolor}
\input{glyphtounicode}


%---------- FONT OPTIONS ----------
% sans-serif
% \usepackage[sfdefault]{FiraSans}
% \usepackage[sfdefault]{roboto}
% \usepackage[sfdefault]{noto-sans}
% \usepackage[default]{sourcesanspro}

% serif
% \usepackage{CormorantGaramond}
% \usepackage{charter}


\pagestyle{fancy}
\fancyhf{} % clear all header and footer fields
\fancyfoot{}
\renewcommand{\headrulewidth}{0pt}
\renewcommand{\footrulewidth}{0pt}

% Adjust margins
\addtolength{\oddsidemargin}{-0.5in}
\addtolength{\evensidemargin}{-0.5in}
\addtolength{\textwidth}{1in}
\addtolength{\topmargin}{-.5in}
\addtolength{\textheight}{1.0in}

\urlstyle{same}

\raggedbottom
\raggedright
\setlength{\footskip}{4.5pt}
\setlength{\tabcolsep}{0in}

% Sections formatting
\titleformat{\section}{
  \vspace{-4pt}\scshape\raggedright\large
}{}{0em}{}[\color{black}\titlerule \vspace{-5pt}]

% Ensure that generate pdf is machine readable/ATS parsable
\pdfgentounicode=1

%-------------------------
% Custom commands

\newcommand{\resumeItem}[1]{
  \item\small{
    {#1 \vspace{-2pt}}
  }
}


\newcommand{\resumeSubheading}[4]{
  \vspace{-2pt}\item
    \begin{tabular*}{0.97\textwidth}[t]{l@{\extracolsep{\fill}}r}
      \textbf{#1} & #2 \\
      \textit{\small#3} & \textit{\small #4} \\
    \end{tabular*}\vspace{-7pt}
}


\newcommand{\resumeSubSubheading}[2]{
    \vspace{-2pt}\item
    \begin{tabular*}{0.97\textwidth}{l@{\extracolsep{\fill}}r}
      \textit{\small#1} & \textit{\small #2} \\
    \end{tabular*}\vspace{-7pt}
}


\newcommand{\resumeEducationHeading}[4]{
  \vspace{-2pt}\item
    \begin{tabular*}{0.97\textwidth}[t]{l@{\extracolsep{\fill}}r}
      \textbf{#1} & #2 \\
      \textit{\small#3} & \textit{\small #4} \\
    \end{tabular*}\vspace{-5pt}
}


\newcommand{\resumeProjectHeading}[2]{
    \vspace{-2pt}\item
    \begin{tabular*}{0.97\textwidth}{l@{\extracolsep{\fill}}r}
      \small#1 & #2 \\
    \end{tabular*}\vspace{-7pt}
}


\newcommand{\resumeOrganizationHeading}[4]{
  \vspace{-2pt}\item
    \begin{tabular*}{0.97\textwidth}[t]{l@{\extracolsep{\fill}}r}
      \textbf{#1} & \textit{\small #2} \\
      \textit{\small#3}
    \end{tabular*}\vspace{-7pt}
}

\newcommand{\resumeSubItem}[1]{\resumeItem{#1}\vspace{-4pt}}

\renewcommand\labelitemii{$\vcenter{\hbox{\tiny$\bullet$}}$}

\newcommand{\resumeSubHeadingListStart}{\begin{itemize}[leftmargin=0.15in, label={}]}
\newcommand{\resumeSubHeadingListEnd}{\end{itemize}}
\newcommand{\resumeItemListStart}{\begin{itemize}}
\newcommand{\resumeItemListEnd}{\end{itemize}\vspace{-5pt}}

%-------------------------------------------
%%%%%%  CV COMMENCE ICI  %%%%%%%%%%%%%%%%%%%%%%%%%%%%


\begin{document}

%---------- PHOTO ----------

\begin{tikzpicture}[remember picture,overlay]
  \node[
    anchor=north west, inner sep=-9pt, draw=gray, line width=1.2pt, circle
  ] at ([xshift=1cm, yshift=-1.5cm]current page.north west)
  {
    \includegraphics[width=2.5cm, height=2.5cm, keepaspectratio]{profile_cv.png}
  };
\end{tikzpicture}

%---------- TITRE ----------

\begin{center}
    \textbf{\Huge \scshape Abdullah Ozer} \\ \vspace{5pt}
    {\LARGE \scshape Ingénieur Test Automation} \\ \vspace{3pt}
    \small
    \faMobile \hspace{.5pt} \href{tel:33769929851}{+337 69 92 98 51}
    $|$
    \faAt \hspace{.5pt} \href{
      mailto:abdullahozer11@hotmail.com
    }{abdullahozer11@hotmail.com}
    $|$
    \faLinkedinSquare \hspace{.5pt} \href{
      https://www.linkedin.com/in/abdullah-ozer-a23733107/?locale=fr
    }{LinkedIn}
    $|$
    \faGithub \hspace{.5pt} \href{https://github.com/abdullahozer11}{GitHub}
    $|$
    \faMapMarker \hspace{.5pt} \href{
      https://www.google.com/maps/place/Paris/@48.8588255,2.2646353,12z/data=!3m1!4b1!4m6!3m5!1s0x47e66e1f06e2b70f:0x40b82c3688c9460!8m2!3d48.856614!4d2.3522219!16zL20vMDVxdGo?entry=ttu
    }{Paris, France}

\raggedright
    \vspace{10pt}
    Fort de plus de 7 ans d'expérience dans le domaine du test logiciel, je suis un ingénieur test
    automation spécialisé dans l'automatisation des tests de protocoles réseau complexes (IPSec,
    MPLS, BGP) et le développement d'infrastructures de test en Python. Je cherche à m'investir dans
    des projets technologiques innovants où je pourrai contribuer à la livraison rapide de solutions
    fiables aux clients à travers la création de frameworks de test sophistiqués et l'optimisation
    des processus d'automatisation.
\end{center}

%----------- DESCRIPTION -----------


%----------- EXPÉRIENCE PROFESSIONNELLE -----------

\section{Expérience Professionnelle}
  \vspace{3pt}
  \resumeSubHeadingListStart

    \resumeSubheading
      {6WIND}{Paris, France}
      {Ingénieur Test Automation (3 ans)}
      {Août 2021 \textbf{--} Présent, CDI}
        \resumeItemListStart
            \resumeItem{
              Conception et implémentation d'un framework d'automatisation de tests en Python pour la validation de logiciels réseau, améliorant significativement la couverture des tests.
            }
            \resumeItem{
              Développement de scénarios de test complexes simulant des infrastructures clients avec des protocoles réseau avancés (IPSec, MPLS, BGP, VLAN, VXLAN, GRE).
            }
            \resumeItem{
              Création et maintenance de suites de tests de performance automatisés pour valider les KPIs réseau, notamment le débit, la latence et le traitement des paquets.
            }
            \resumeItem{
              Mise en place de pipelines CI/CD et développement d'outils de test personnalisés en Python pour optimiser le processus de test.
            }
        \resumeItemListEnd

    \resumeSubheading
      {STMicroelectronics}{Grenoble, France}
      {Ingénieur Développement Test (2 ans)}
      {Août 2019 \textbf{--} Août 2021, CDI}
        \resumeItemListStart
            \resumeItem{
              Développement d'outils d'automatisation de tests en Python pour la validation de composants RF, intégration avec divers équipements de mesure de signal.
            }
            \resumeItem{
              Création de scripts de test automatisés pour la validation matérielle et les tests de performance.
            }
            \resumeItem{
              Implémentation de bibliothèques Python personnalisées pour le contrôle des équipements de test et l'analyse des données.
            }
        \resumeItemListEnd

    \resumeSubheading
      {Airties Wireless Networks}{Istanbul, Turquie}
      {Ingénieur Test (2 ans)}
      {Fév. 2016 \textbf{--} Fév. 2018, CDI}
        \resumeItemListStart
            \resumeItem{
              Conception de solutions de test automatisées en Python pour les équipements réseau.
            }
            \resumeItem{
              Développement d'outils d'automatisation pour le contrôle matériel, incluant des stations de charge automatisées et des plateformes de test intelligentes.
            }
            \resumeItem{
              Création de scripts Python pour les tests de régression automatisés et la validation des performances.
            }
        \resumeItemListEnd

  \resumeSubHeadingListEnd

%----------- COMPÉTENCES -----------

\section{Compétences}
  \vspace{2pt}
  \resumeSubHeadingListStart
    \small{\item{
        \textbf{Langages de Programmation:}{
          Python 3.6+, JavaScript (ES6), TypeScript, Bash, HTML 5, CSS 3, PostgreSQL
        } \\ \vspace{3pt}
        \textbf{Développement Python:}{
          Django, Django REST Framework, Unittest
        } \\ \vspace{3pt}
        \textbf{Protocoles Réseau:}{
          IPSec, MPLS, BGP, VLAN, VXLAN, GRE, TCP/IP
        } \\ \vspace{3pt}
        \textbf{Outils de Test:}{
          Frameworks de Test Python, Analyseurs Réseau, Outils CI/CD
        } \\ \vspace{3pt}
        \textbf{Langues:}{
          Anglais (courant), Français (courant), Turc (langue maternelle)
        } \\ \vspace{3pt}
    }}
  \resumeSubHeadingListEnd

%----------- FORMATION -----------

\section{Formation}
  \vspace{3pt}
  \resumeSubHeadingListStart

    \resumeEducationHeading
      {ESIGELEC}
      {Rouen, France}
      {Master of Science \textbf{--} Spécialité: Systèmes Électroniques Embarqués}
      {Fév. 2018 \textbf{--} Août 2019}

    \resumeEducationHeading
      {Université du Bosphore (Boğaziçi Üniversitesi)}
      {Istanbul, Turquie}
      {Diplôme d'Ingénieur \textbf{--} Spécialité: Génie Électrique et Électronique}
      {Fév. 2011 \textbf{--} Août 2016}

  \resumeSubHeadingListEnd

%----------- PROJETS -----------

\section{Projets}
    \vspace{3pt}
    \resumeSubHeadingListStart

      \resumeProjectHeading
        {\textbf{Framework de Test des Protocoles Réseau} $|$ \emph{Python, PostgreSQL}}{}
          \resumeItemListStart
            \resumeItem{
              Développement d'un framework complet en Python pour tester des protocoles et scénarios réseau complexes.
            }
            \resumeItem{
              Implémentation de suites de tests automatisés pour la validation des tunnels IPSec, chemins MPLS et configurations BGP.
            }
            \resumeItem{
              Création d'outils personnalisés de reporting et d'analyse utilisant PostgreSQL pour la gestion des résultats de test.
            }
          \resumeItemListEnd

      \resumeProjectHeading
        {\textbf{Infrastructure de Test Automatisée} $|$ \emph{Python, Bash}}{}
          \resumeItemListStart
            \resumeItem{
              Construction d'une infrastructure de test automatisée pour la validation des performances réseau.
            }
            \resumeItem{
              Développement de scripts Python pour l'automatisation de la configuration et du démontage des environnements de test.
            }
            \resumeItem{
              Création de systèmes de surveillance et de journalisation pour les suites de tests longue durée.
            }
          \resumeItemListEnd

    \resumeSubHeadingListEnd

\end{document}
