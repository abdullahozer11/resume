%-------------------------
% CV in Latex
% Auteur : Abdullah Ozer
% License : MIT
%------------------------

\documentclass[letterpaper,10pt]{article}

\usepackage{bold-extra}
\usepackage{latexsym}
\usepackage[empty]{fullpage}
\usepackage{titlesec}
\usepackage{marvosym}
\usepackage[usenames,dvipsnames]{color}
\usepackage{verbatim}
\usepackage{enumitem}
\usepackage[hidelinks]{hyperref}
\usepackage{fancyhdr}
\usepackage{graphicx}
\usepackage[english]{babel}
\usepackage{tabularx}
\usepackage{hyphenat}
\usepackage{fontawesome}
\usepackage{tikz}
\usepackage{graphicx}
\usepackage{xcolor}
\input{glyphtounicode}


%---------- FONT OPTIONS ----------
% sans-serif
% \usepackage[sfdefault]{FiraSans}
% \usepackage[sfdefault]{roboto}
% \usepackage[sfdefault]{noto-sans}
% \usepackage[default]{sourcesanspro}

% serif
% \usepackage{CormorantGaramond}
% \usepackage{charter}


\pagestyle{fancy}
\fancyhf{} % clear all header and footer fields
\fancyfoot{}
\renewcommand{\headrulewidth}{0pt}
\renewcommand{\footrulewidth}{0pt}

% Adjust margins
\addtolength{\oddsidemargin}{-0.5in}
\addtolength{\evensidemargin}{-0.5in}
\addtolength{\textwidth}{1in}
\addtolength{\topmargin}{-.5in}
\addtolength{\textheight}{1.0in}

\urlstyle{same}

\raggedbottom
\raggedright
\setlength{\footskip}{4.5pt}
\setlength{\tabcolsep}{0in}

% Sections formatting
\titleformat{\section}{
  \vspace{-4pt}\scshape\raggedright\large
}{}{0em}{}[\color{black}\titlerule \vspace{-5pt}]

% Ensure that generate pdf is machine readable/ATS parsable
\pdfgentounicode=1

%-------------------------
% Custom commands

\newcommand{\resumeItem}[1]{
  \item\small{
    {#1 \vspace{-2pt}}
  }
}


\newcommand{\resumeSubheading}[4]{
  \vspace{-2pt}\item
    \begin{tabular*}{0.97\textwidth}[t]{l@{\extracolsep{\fill}}r}
      \textbf{#1} & #2 \\
      \textit{\small#3} & \textit{\small #4} \\
    \end{tabular*}\vspace{-7pt}
}


\newcommand{\resumeSubSubheading}[2]{
    \vspace{-2pt}\item
    \begin{tabular*}{0.97\textwidth}{l@{\extracolsep{\fill}}r}
      \textit{\small#1} & \textit{\small #2} \\
    \end{tabular*}\vspace{-7pt}
}


\newcommand{\resumeEducationHeading}[4]{
  \vspace{-2pt}\item
    \begin{tabular*}{0.97\textwidth}[t]{l@{\extracolsep{\fill}}r}
      \textbf{#1} & #2 \\
      \textit{\small#3} & \textit{\small #4} \\
    \end{tabular*}\vspace{-5pt}
}


\newcommand{\resumeProjectHeading}[2]{
    \vspace{-2pt}\item
    \begin{tabular*}{0.97\textwidth}{l@{\extracolsep{\fill}}r}
      \small#1 & #2 \\
    \end{tabular*}\vspace{-7pt}
}


\newcommand{\resumeOrganizationHeading}[4]{
  \vspace{-2pt}\item
    \begin{tabular*}{0.97\textwidth}[t]{l@{\extracolsep{\fill}}r}
      \textbf{#1} & \textit{\small #2} \\
      \textit{\small#3}
    \end{tabular*}\vspace{-7pt}
}

\newcommand{\resumeSubItem}[1]{\resumeItem{#1}\vspace{-4pt}}

\renewcommand\labelitemii{$\vcenter{\hbox{\tiny$\bullet$}}$}

\newcommand{\resumeSubHeadingListStart}{\begin{itemize}[leftmargin=0.15in, label={}]}
\newcommand{\resumeSubHeadingListEnd}{\end{itemize}}
\newcommand{\resumeItemListStart}{\begin{itemize}}
\newcommand{\resumeItemListEnd}{\end{itemize}\vspace{-5pt}}

%-------------------------------------------
%%%%%%  CV COMMENCE ICI  %%%%%%%%%%%%%%%%%%%%%%%%%%%%


\begin{document}

%---------- PHOTO ----------

\begin{tikzpicture}[remember picture,overlay]
  \node[
    anchor=north west, inner sep=-9pt, draw=gray, line width=1.2pt, circle
  ] at ([xshift=1cm, yshift=-1.5cm]current page.north west)
  {
    \includegraphics[width=2.5cm, height=2.5cm, keepaspectratio]{profile_cv.png}
  };
\end{tikzpicture}

%---------- TITRE ----------

\begin{center}
    \textbf{\Huge \scshape Abdullah Ozer} \\ \vspace{5pt}
    {\LARGE \scshape Développeur Logiciel} \\ \vspace{3pt}
    \small
    \faMobile \hspace{.5pt} \href{tel:33769929851}{+337 69 92 98 51}
    $|$
    \faAt \hspace{.5pt} \href{
      mailto:abdullahozer11@hotmail.com
    }{abdullahozer11@hotmail.com}
    $|$
    \faLinkedinSquare \hspace{.5pt} \href{
      https://www.linkedin.com/in/abdullah-ozer-a23733107/?locale=fr
    }{LinkedIn}
    $|$
    \faGithub \hspace{.5pt} \href{https://github.com/abdullahozer11}{GitHub}
    $|$
    \faMapMarker \hspace{.5pt} \href{
      https://www.google.com/maps/place/Paris/@48.8588255,2.2646353,12z/data=!3m1!4b1!4m6!3m5!1s0x47e66e1f06e2b70f:0x40b82c3688c9460!8m2!3d48.856614!4d2.3522219!16zL20vMDVxdGo?entry=ttu
    }{Paris, France}

    \raggedright
    \vspace{10pt}
    Ayant plus de 5 ans d'expérience dans le développement logiciel, je suis à
    la recherche d'un poste de développeur web. Mon expertise inclut la conception
    d'API, l'intégration de systèmes, ainsi que la création rapide de frontends en
    utilisant React et ses frameworks dérivés.
\end{center}


%----------- DESCRIPTION -----------

%----------- PROJETS -----------

\section{Projets}
    \vspace{3pt}
    \resumeSubHeadingListStart

      \resumeProjectHeading
        {
          \textbf{
            SplitFree (Typescript, Javascript, Django, Python, PostgreSQL, Deno)
          } $|$ \emph{
            \href{
              https://github.com/abdullahozer11/split-free-frontend
            }{GitHub}
          }
        }{}
          \resumeItemListStart
            \resumeItem{
              Construit une application intelligente et gratuite de répartition des dépenses pour mobile en utilisant React et Supabase.
            }
            \resumeItem{
              Utilisé React Query pour collecter et mettre en cache les données efficacement.
            }
            \resumeItem{
              Utilisé NativeWind pour créer des interfaces utilisateur belles et réactives.
            }
            \resumeItem{
              Utilisé l'API Gemini pour classer automatiquement les dépenses des utilisateurs.
            }
            \resumeItem{
              Créé un backend Django pour gérer les calculs algorithmiques afin de présenter
              des solutions de paiement aux utilisateurs finaux.
            }
            \resumeItem{
              Utilisé Poetry pour la gestion des paquets. Dockerisé pour une intégration facile.
              Créé des actions GitHub pour CI/CD.
            }
          \resumeItemListEnd

      \resumeProjectHeading
        {
          \textbf{
            Site Portfolio (Python, Django, Unittest, Docker, React, Javascript, GSAP)
          } $|$ \emph{
            \href{
              https://github.com/abdullahozer11/MasterWebsite
            }{GitHub}
          }
        }{}
          \resumeItemListStart
            \resumeItem{
              Créé un site web de portfolio personnel en utilisant des animations avec GSAP.
            }
            \resumeItem{
              Créé un jeu de devinettes de mots utilisant des cylindres hexagonaux pour former
              un maximum de mots en faisant des vérifications via Django REST framework et un
              dictionnaire anglais open source.
            }
            \resumeItem{
              Présenté des projets de démonstration tels que des minuteurs personnalisés pour
              couples pour leurs retrouvailles en utilisant Midjourney, HTML et Photoshop.
            }
          \resumeItemListEnd

    \resumeSubHeadingListEnd


%----------- ÉDUCATION -----------

\section{Éducation}
  \vspace{3pt}
  \resumeSubHeadingListStart

    \resumeEducationHeading
      {ESIGELEC}
      {Rouen, France}
      {Master en Sciences et Technologies \textbf{--} Spécialité: Systèmes Électroniques Embarquées}
      {Fev 2018 \textbf{--} Aout 2019}

    \resumeEducationHeading
      {Bogazici University}
      {Istanbul, Türkiye}
      {License \textbf{--} Filière: Electric and Electronics Engineering}
      {Fev 2011 \textbf{--} Aout 2016}

  \resumeSubHeadingListEnd

%----------- COMPÉTENCES -----------


\section{Compétences}
  \vspace{2pt}
  \resumeSubHeadingListStart
    \small{\item{
        \textbf{Languages de programmation:}{
          Python 3.6+, JavaScript (ES6), TypeScript, Bash, HTML 5, CSS 3, PostgreSQL
        } \\ \vspace{3pt}
        \textbf{Développement Python:}{
          Django, Django REST Framework, Unittest
        } \\ \vspace{3pt}
        \textbf{Développement JavaScript:}{
          Node.js, Angular, React, Next, Jest, Tailwind
        } \\ \vspace{3pt}
        \textbf{Développement Cloud \& DevOps:}{
          Git, Jenkins, GitHub, Docker, AWS
        } \\ \vspace{3pt}
        \textbf{Langues:}{
          Français (niveau professionnel), Anglais (niveau professionnel), Turc (langue natale)
        } \\ \vspace{3pt}
    }}
  \resumeSubHeadingListEnd

%----------- EXPÉRIENCES PROFESSIONNELLES -----------

\section{Expériences professionnelles}
  \vspace{3pt}
  \resumeSubHeadingListStart

    \resumeSubheading
      {6WIND}{Paris, France}
      {Développeur Python (3 années)}
      {Aout 2021 \textbf{--} en cours, CDI}

        \resumeItemListStart
          \resumeItem{
            Élargi les capacités de test automatisé des entreprises en ajoutant
            de nouvelles suites de tests, topologies, outils et assistants à leur
            cadre de test.
          }
%          \resumeItem{
%            Amélioré l'expérience des entreprises avec les fonctionnalités liées
%            à la performance en créant de nouvelles façons d'afficher les KPI sur
%            des tests de performance cruciaux.
%          }
          \resumeItem{
            Démontré réactivité et agilité lors des moments critiques où les clients
            avaient besoin de tests rapides imitant leur infrastructure, en fournissant
            des rapports et des métriques adaptés à leurs besoins spécifiques.
          }
        \resumeItemListEnd

    \resumeSubheading
      {STMicroelectronics}{Grenoble, France}
      {Développeur Logiciel (2 années)}
      {Aout 2019 \textbf{--} Aout 2021, CDI}

        \resumeItemListStart
            \resumeItem{
              Développé des outils et des pilotes pour tester des émetteurs/récepteurs
              RF à haute fréquence et courte distance avec une variété de matériel de
              test tel que des atténuateurs de signal.
            }
%            \resumeItem{
%              Développé des logiciels embarqués sur STM32 pour gérer l'entrée de tension
%              pour les équipements de test.
%            }
        \resumeItemListEnd

    \resumeSubheading
      {Airties Wireless Networks}{Istanbul, Türkiye}
      {Ingénieur de Test  (2 années)}
      {Fev 2016 \textbf{--} Fev 2018, CDI}

        \resumeItemListStart
          \resumeItem{
            Développé des équipements d'automatisation cruciaux tels que des stations de recharge
            automatiques et des ascenseurs intelligents pour des projets d'automatisation de tests
            utilisant Arduino, moteurs et capteurs.
          }
%          \resumeItem{
%            Créé des tests automatisés de qualité vidéo et de propension aux erreurs pour les
%            décodeurs (boîtiers TV).
%          }
        \resumeItemListEnd

  \resumeSubHeadingListEnd


\end{document}
